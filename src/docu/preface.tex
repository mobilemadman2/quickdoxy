\thispagestyle{empty}
%\begin{multicols}{2}
\begin{flushleft}
\emph{Preface}
\end{flushleft}


\textit{quickdoxy} provides a graphical wrapper around a portable doxygen distribution. It can be used to extract a copy of that 
distribution on any win64 machine, and to quickly create documentation of existing code. 

\textit{quickdoxy} does not provide means for detailed parameterization of the doxygen run/output. However, any parameterization
done in the config files of the extracted doxygen distribution will be taken into account when running \textit{quickdoxy}.
\begin{flushleft}

-- --

\end{flushleft}

\noindent Within this document, the following keywords and symbols are used to \mbox{emphasise} certain information:
\begin{IMPORTANT}
Information on limitations of the software, background knowledge, and user interactions that may damage existing files on the computer.
\end{IMPORTANT}
\begin{USEFUL}
Insightful information on how to use \textit{quickdoxy} to its full potential.
\end{USEFUL}		
\begin{BUG}
Known bug or issue that must be worked around.
\end{BUG}			
\addtocounter{page}{-1}
\newpage	
%\end{multicols}